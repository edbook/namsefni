\let\oldmarginpar\marginpar
\renewcommand\marginpar[1]{\-\oldmarginpar[\raggedleft\footnotesize \textit{#1}]%
{\raggedright\footnotesize #1}}

% -------------------------------------------------
% Til að setja skjalið í leturgerðina GFS Artemisia
% -------------------------------------------------
\usepackage{gfsartemisia-euler}
\renewcommand\familydefault{\sfdefault}

\usepackage{a4wide}
\usepackage{multicol} % Hægt að kalla á dálka inni í miðju skjali, notað t.d. í æfingum og svarkafla.
\usepackage{subfig} % Nokkrar myndir hlið við hlið, notað í Fleygbogakaflanum
\usepackage{comment}
%\usepackage[margin=1.5cm]{geometry}

% Tungumálapakkar
\usepackage[utf8]{inputenc} % Nauðsynlegt fyrir íslenska stafi.
\usepackage[T1]{fontenc}
\usepackage[icelandic]{babel} % Íslenskt stafasett ofl. Líka fyrir línuskiptingar og slíkt.

% Stærðfræðipakkar
\usepackage{amsmath} % Nauðsynlegt fyrir t.d. align
\usepackage{amsfonts}
\usepackage{amssymb}
\usepackage{calc}
\usepackage{polynom}
\usepackage{cancel}


% Ýmiskonar forsnið
\usepackage{ulem} % Undirstrikanir
\usepackage{setspace} % Gefur möguleika á einu og hálfu eða tvöföldu línubili
\usepackage{pgf,tikz}
\usepackage{pgfplots}
\usepackage{float} % Fyrir H stika í figure
\usetikzlibrary{arrows,positioning}
\usetikzlibrary{shapes}
\usetikzlibrary{shapes.callouts, decorations.text} % LATEX and plain TEX when using Tik Z
\usetikzlibrary{calc}                                                       % ATH. bætt við fyrir mynd í kafla um útgildi
\usepackage[explicit]{titlesec} % Til að gera fyrirsagnir
\usepackage{fancyhdr} % Til að gera fyrirsagnir
\usepackage{changepage} % Til að breyta spássíu á svörum
\usepackage{units} % Forsníður einingar
\usepackage{etoolbox}
\usepackage{enumerate}
%\usepackage{ifthen}
\usepackage{mdframed}

\pgfplotsset{compat=1.12}

\usepackage{rotating}

\usepackage[pdftex, %
pdfauthor={Hólmfríður Þorsteinsdóttir, Valdís Björk Þorsteinsdóttir og Jóhann Sigursteinn Björnsson}, %
pdftitle={Heildun}%
]{hyperref}



% ----------------------------------------------------------------
% Spássíur og bil
% ----------------------------------------------------------------

\newlength{\hanging} % Skilgreini \hanging sem hangandi spássíu, til að nota í síðari skilgreiningum
\newlength{\hangingsvor} % Skilgreini \hanging sem hangandi spássíu, til að nota í síðari skilgreiningum
\newlength{\hangingaefingar} % Skilgreini \hanging sem hangandi spássíu, til að nota í síðari skilgreiningum

 

\setlength{\hanging}{0cm} % Breyta þessari tölu ef breyta á hangangi spássíu. Þá breytist hún allstaðar í bókinni
\setlength{\hangingsvor}{0cm} % Breyta þessari tölu ef breyta á hangangi spássíu. Þá breytist hún allstaðar í bókinni
\setlength{\hangingaefingar}{0cm} % Breyta þessari tölu ef breyta á hangangi spássíu. Þá breytist hún allstaðar í bókinni

 

\newlength{\horn} % Skilgreini \hanging sem hangandi spássíu, til að nota í síðari skilgreiningum
\setlength{\horn}{0.5cm} % Breyta þessari tölu ef breyta á hangangi spássíu. Þá breytist hún allstaðar í bókinni

\setlength{\parskip}{10pt} % Bil milli efnisgreina
\setlength{\parindent}{0cm} % Inndráttur fyrstu línu
\setlength{\oddsidemargin}{\hanging} % Vinstri spássía á oddatölusíðum
\setlength{\evensidemargin}{\hanging} % Vinstri spássía á sléttratölusíðum
\setlength{\textwidth}{\textwidth-\hanging} % Breidd texta er stöðluð breidd texta að frádreginni hangandi spássíu
\setlength{\textheight}{\textheight+\topmargin+\voffset+\headsep+1cm}
\setlength{\topmargin}{0cm}
\setlength{\voffset}{0cm}
\setlength{\headsep}{0.5cm}
\setlength{\headheight}{15.2pt}

%\setlength{\topmargin}{0cm}

\setlength{\marginparwidth}{\hanging}

\newlength{\bileftirreglu}
\setlength{\bileftirreglu}{0.5cm}
\newlength{\bilfyrirreglu}
\setlength{\bilfyrirreglu}{0.5cm}
\newlength{\bileftirsyn}
\setlength{\bileftirsyn}{0.3cm}
\newlength{\bilfyrirsyn}
\setlength{\bilfyrirsyn}{0.3cm}

\let\stdsection\section
\renewcommand\section{\newpage\stdsection}


% --- Litir fyrir skjalið ------------------------------------------------
\definecolor{reglulitur}{rgb}{0.7,1,0.7} % Grænn
\definecolor{hreglulitur}{rgb}{1,0.65,0} % Appelsínugulur
\definecolor{skilgrlitur}{rgb}{0.75,0.75,1} % Blár

\definecolor{ljosgrar}{rgb}{0.85,0.85,0.85}
\definecolor{dokkgrar}{rgb}{0.25,0.25,0.25}
\definecolor{dokkgraenn}{rgb}{0,0.75,0}

% --- Stílar fyrir myndir ------------------------------------------------
\tikzstyle{mynd} = [line cap=round,line join=round,>=triangle 45, baseline={(0,0)}]
\tikzstyle{asar} = [color=black, ->]
\tikzstyle{xas} = [color=black, font=\footnotesize, below, anchor=north, shift={(0,-4pt)}]
\tikzstyle{xkvardi} = [color=black, font=\footnotesize, below]
\tikzstyle{yas} = [color=black, font=\footnotesize, left, anchor=east]
\tikzstyle{ykvardi} = [color=black, font=\footnotesize, left]
\tikzstyle{punktur} = [color=blue, font=\small]
\tikzstyle{texti} = [color=black, font=\footnotesize]
\tikzstyle{ferill} = [color=blue, smooth, samples=101]
\tikzstyle{ferill2} = [color=dokkgraenn, smooth, samples=101]
\tikzstyle{lina} = [color=black, smooth, samples=100]
\tikzstyle{punktalina} = [color=gray, dashed]
\tikzstyle{or} = [dashed, ->, color=green!70!black]
\tikzstyle{rudur} = [color=ljosgrar,dashed, xstep=0.5cm, ystep=0.5cm]

\tikzstyle{bil} = [line width=1.6pt]
\tikzstyle{lokad} = [draw=black, fill=black]
\tikzstyle{opid} = [draw=black, fill=white]
\tikzstyle{rot} = [color=black, font=\footnotesize, above]
\tikzstyle{formerki}= [color=black, font=\footnotesize, anchor=base, shift={(0,-13pt)}]
\tikzstyle{formerki_staerd}= [color=black, font=\footnotesize, shift={(-1.3ex,-13pt)}, anchor=base west]
\tikzstyle{formerki_staerd2}= [color=black, font=\footnotesize, shift={(-1.3ex,-26pt)}, anchor=base west]
\tikzstyle{upp} = [color=black, font=\footnotesize, anchor=base, shift={(0,-26pt)}]
%\tikzstyle{niður} = [color=black, font=\footnotesize, anchor=base, shift={(0,-26pt)}]
%\tikzstyle{sveigja_u} = [color=black, font=\footnotesize, anchor=base, shift={(0,-26pt)}]
%\tikzstyle{sveigja_n} = [color=black, font=\footnotesize, anchor=base, shift={(0,-26pt)}]

\tikzstyle{texti} = [font=\footnotesize]
\tikzstyle{form} = [fill=green!40, draw=black]
\tikzstyle{skyring} = [fill=green!40, fill opacity=0.7]

\pagestyle{fancy}

\fancyhfoffset[L]{\hanging}

\newcounter{regluteljari}[section]
\renewcommand{\theregluteljari}{\thesection.\arabic{regluteljari}}

\newenvironment{regla}[1]
{\addtocounter{regluteljari}{-1}\refstepcounter{regluteljari}\par
\begin{tikzpicture}
\tikzstyle{regla}=[fill=reglulitur, text width=\textwidth-4ex, inner sep=2ex]
\node[regla, anchor=north west] (0,0) \bgroup\textbf{Regla \theregluteljari\ #1}\par
}
{\egroup;\end{tikzpicture}\stepcounter{regluteljari}}

\newenvironment{hregla}[1]
{\addtocounter{regluteljari}{-1}\refstepcounter{regluteljari}\par
\begin{tikzpicture}
\tikzstyle{hregla}=[fill=hreglulitur, text width=\textwidth-4ex, inner sep=2ex]
\node[hregla, anchor=north west] (0,0) \bgroup\textbf{Hjálparregla \theregluteljari\ #1}\par
}
{\egroup;\end{tikzpicture}\stepcounter{regluteljari}}

\newenvironment{sonnun}
{\textbf{Sönnun:}\linebreak}
{\par\hfill$\blacksquare$}

\newenvironment{monnun}
{\textbf{Myndrænt:}\linebreak}
{\par\hfill$\blacksquare$}

\newenvironment{onnun}
{\textbf{{Önnur\,\,sönnun:} \hfill}\linebreak}
{\par\hfill$\blacksquare$}

\newcounter{skilgrteljari}[section]
\renewcommand{\theskilgrteljari}{\thesection.\arabic{skilgrteljari}}
\newenvironment{skilgr}[1]
{\addtocounter{skilgrteljari}{-1}\refstepcounter{skilgrteljari}\par
\begin{tikzpicture}[remember picture]
\tikzstyle{skilgr}=[fill=skilgrlitur, text width=\textwidth-4ex, inner sep=2ex]
\tikzstyle{lykill}=[text width=\hanging-4ex, inner sep=2ex, text centered]
\node[anchor=north east, lykill] (0,0) {\textit{#1}};
\node[anchor=north west,skilgr] (0,0)\bgroup\textbf{Skilgreining \theskilgrteljari\ #1}
\par
}
{\egroup;\end{tikzpicture}\stepcounter{skilgrteljari}}

\newenvironment{bluebox}[1]
{\begin{tikzpicture}[remember picture]
\tikzstyle{bluebox}=[fill=skilgrlitur, text width=\textwidth-4ex, inner sep=2ex]
\tikzstyle{lykill}=[text width=\hanging-4ex, inner sep=2ex, text centered]
\node[anchor=north east, lykill] (0,0) {\textit{}};
\node[anchor=north west,bluebox] (0,0)\bgroup\textbf{}
\par
}
{\egroup;\end{tikzpicture}}

\newcounter{frumsenduteljari}[section]
\renewcommand{\thefrumsenduteljari}{\thesection.\arabic{frumsenduteljari}}
\newenvironment{frumsenda}[1]
{\par
\begin{tikzpicture}[remember picture]
\tikzstyle{frumsenda}=[fill=ljosgrar, text width=\textwidth-4ex, inner sep=2ex]
\tikzstyle{lykill}=[text width=\hanging-4ex, inner sep=2ex, text centered]
\node[anchor=north east, lykill] (0,0) {\textit{#1}};
\node[anchor=north west,frumsenda] (0,0)\bgroup\textbf{Frumsenda #1}
\par
}
{\egroup;\end{tikzpicture}\stepcounter{frumsenduteljari}}

\newenvironment{syn}[1]
{
\vspace{\bilfyrirsyn}
\hspace{-\horn}\begin{tikzpicture}%[remember picture]\hspace{-\hanging}
\tikzstyle{horn}=[inner sep=0ex, text width=\horn] %, align=right,text width=\hanging-0ex,
\node[anchor=north east, horn] (0,0) {\LARGE$\ulcorner$\normalsize};
\end{tikzpicture}\textbf{Sýnidæmi um #1}
}
{
\vspace{-0.3cm} %\linebreak
\hspace{-\horn}\begin{tikzpicture}[remember picture, baseline={(0,0)}]
\tikzstyle{horn}=[text width=\horn, inner sep=0ex] % , align= right
\node[anchor=south east, horn] (0,0) {\LARGE$\llcorner$\normalsize};
\end{tikzpicture}
\vspace{\bileftirsyn}
}


%\newenvironment{syn}[1]
% {\vspace{\bilfyrirsyn}\textbf{Sýnidæmi um #1}\marginpar{\hfill\LARGE$\ulcorner$\normalsize}\par}
% {\marginpar{\hfill\vspace*{2mm}\LARGE$\llcorner$}\normalsize\vspace{\bileftirsyn}}

\newenvironment{lausn}
{\par\underline{Lausn}\par}
{\par}
\newenvironment{lausnb} % Lausn sem byrjar á formúlu
{\par\underline{Lausn}}
{\unskip}
\newenvironment{lausne} % Lausn sem endar á formúlu
{\par\underline{Lausn}\par}
{\unskip\vspace{-1cm}}
\newenvironment{lausnbe} % Lausn sem byrjar og endar á formúlu
{\par\underline{Lausn}}
{\unskip\vspace{-1cm}}

\newenvironment{ath} % Athugasemd
{\par\underline{Athugasemd}\par}
{\par}

\newenvironment{æd} % Æfingardæmi
{\par\underline{Æfingardæmi}\par}
{\par}

\newcommand{\lokasvar}[1]{\underline{#1}}


\begin{comment}
\titleformat{\section}
{\Large\bfseries}
{\gdef\chapterlabel{\thesection\ }}{0pt}
{\setcounter{regluteljari}{1}\begin{tikzpicture}[remember picture,overlay]
\fill[gray!10] (0,0) rectangle (\textwidth+3cm,20pt);
\node[anchor=text,rectangle,
rounded corners=20pt,inner sep=11pt]
{\color{blue!50!black}\chapterlabel#1};
\end{tikzpicture}
}
\end{comment}
\titlespacing*{\section}{-\hanging}{25pt}{0pt} % {Nafn titils}{Vinstra bil}{Fyrir ofan}{Fyrir neðan}
\titlespacing*{\subsection}{-\hanging}{20pt}{5pt}

\titlespacing*{\subsubsection}{-\hanging}{0pt}{0pt}